%!TeX program = xelatex
%Do not change
\documentclass[12pt, oneside]{article}
\usepackage{amssymb,amsmath}
\usepackage[margin=1in]{geometry}
\usepackage{textpos}
\usepackage{float}
%\usepackage{color}
\usepackage{graphicx}
\usepackage[inter-unit-product =\cdot]{siunitx}
%\usepackage{tikz}
%\usetikzlibrary{positioning}
%\usepackage{tikz-3dplot}
%\usepackage{pgfopts}
%\usepackage{wasysym}
%\usepackage{stanli}

% You may add the packages you need here
\begin{document}

\begin{center}
\textbf{\Large Project 2}

\textbf{Due 2 April 2021}
\end{center}

Before adding a leg vise to his workbench, Dr. Smith wants to better understand some of the stresses and deflections.
This project will build off the analysis already performed in Project 1, for those who desire, analysis from Project 1 may be repeated to recover some of the points lost.
%recover points from project 1
\begin{itemize}
	\item Project 1 point recovery
		\begin{itemize}
			\item Assume Dr. Smith applies a force of 20 lbs at the end of the screw handle.
			\item You may choose appropriate dimensions and parameters as needed (for example: handle length and screw pitch). 
			\item \textbf{Hint:} Use simple physics principles (i.e. work done, mechanical advantage) to determine the force in the screw exerted. Formulas that you find via Google will be needlessly complicated (friction effects may be neglected).
			\item Relate the force applied on the handle to force in the vise screw (up to 10 points recovery possible) 
			\item Perform statics analysis on the vise chop with a St. Peter's Cross to find forces in the vise screw, workpiece, and in the St. Peter's cross mechanism (up to 30 points recovery possible)
		\end{itemize}
	\item When fully loaded, how far does the vise screw stretch due to axial loading (25 points)
	\item When fully loaded, how far does the vise screw twist due to torsion? (25 points)
	\item When fully loaded, what is the maximum bending stress in the St. Peter's Cross mechanism (25 points)
	\item Do the stresses you have found seem reasonable for your chosen materials? What could be done to mitigate any problematic stresses? (15 points)
	\item Your overall presentation quality (legibility, professionalism, grammar) will also be worth 10 points
\end{itemize}
%scoring rubric
%axial stretch in loaded screw
%torsional twist in loaded screw
%bending stress st. peter's cross

\end{document}
