%!TeX program = xelatex
%Do not change
\documentclass[12pt, oneside]{article}
\usepackage{amssymb,amsmath}
\usepackage[margin=1in]{geometry}
\usepackage{textpos}
\usepackage{float}
%\usepackage{color}
\usepackage{graphicx}
\usepackage[inter-unit-product =\cdot]{siunitx}
%\usepackage{tikz}
%\usetikzlibrary{positioning}
%\usepackage{tikz-3dplot}
%\usepackage{pgfopts}
%\usepackage{wasysym}
%\usepackage{stanli}

% You may add the packages you need here
\begin{document}

\begin{center}
\textbf{\Large Project 3}

\textbf{Due 20 Nov 2020}
\end{center}

Dr. Smith is now convinced he wants to install a St. Peter's Cross mechanism with his leg vise, but he wants to size the pieces to limit the bending deflection.
This project will build off the analysis already performed in Projects 1 and 2, for those who desire, past analysis may be repeated to recover some of the points lost.
%recover points from project 1
\begin{itemize}
	\item Past project point recovery
		\begin{itemize}
			\item Assume Dr. Smith applies a force of 20 lbs at the end of the screw handle.
			\item You may choose appropriate dimensions and parameters as needed (for example: handle length and screw pitch). 
			\item \textbf{Hint:} Use simple physics principles (i.e. work done, mechanical advantage) to determine the force in the screw exerted. Formulas that you find via Google will be needlessly complicated (friction effects may be neglected).
			\item Relate the force applied on the handle to force in the vise screw (up to 10 points recovery possible) 
			\item Perform statics analysis on the vise chop with a St. Peter's Cross to find forces in the vise screw, workpiece, and in the St. Peter's cross mechanism (up to 20 points recovery possible)
			\item Calculate the bending moment in the St. Peter's cross mechanism (up to 15 points recovery possible)
		\end{itemize}
	\item Design the St. Peter's Cross members such that under the loading case designed for they deflect no more than 0.5 in.
		Compare at least three different beam designs (cross-sections) in your analysis (70 points)
	\item Do the stresses and displacements you have found seem reasonable for your chosen materials? What surprised you in your analysis? (20 points)
	\item Your overall presentation quality (legibility, professionalism, grammar) will also be worth 10 points
\end{itemize}

\end{document}
