%!TeX program = xelatex
%Do not change
\documentclass[12pt, oneside]{article}
\usepackage{amssymb,amsmath}
\usepackage[margin=1in]{geometry}
\usepackage{textpos}
\usepackage{float}
%\usepackage{color}
\usepackage{graphicx}
\usepackage[inter-unit-product =\cdot]{siunitx}
%\usepackage{tikz}
%\usetikzlibrary{positioning}
%\usepackage{tikz-3dplot}
%\usepackage{pgfopts}
%\usepackage{wasysym}
%\usepackage{stanli}

% You may add the packages you need here
\begin{document}

%TODO change numbers in problems
\begin{textblock*}{4cm}(-1.7cm,-2.3cm)
\noindent {\scriptsize AE333 Fall 2020}
\end{textblock*}

%Do not modify other than putting your name where stated
\begin{textblock*}{8cm}(12.5cm,-1cm)
\noindent {Name: }
\end{textblock*}
%Do not modify other than typing your acknowledgement where stated
\begin{textblock*}{13.5cm}(-1.7cm,-1.8cm)
%\noindent \textit{\footnotesize Acknowledgement: Your acknowledgement for collaboration and other sources goes here. }
\end{textblock*}

\vspace{1cm}

%Do not modify other than typing the homework number after #
\begin{center}
\textbf{\Large Homework 1}

\textbf{Due 1 Sep 2020}
\end{center}

%Rest should contain your solution for the homework. Feel free to improvise in ways that you believe make grading easier.
\begin{enumerate}
	\item %tension strain problem

	\item %tension strain problem

	\item %shear strain problem

	\item %shear strain problem

	\item %elastic mechanical properties

	\item %elastic mechanical properties

	\item %plastic mechanical properties

	\item %plastic mechanical properties

	\item %how much strain in rubber band stretched over a rubber band gun

	\item %delta L for vise screw (compare steel and aluminum)
\end{enumerate}
\end{document}
